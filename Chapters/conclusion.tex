\chapter{Conclusions}
\label{ch:conclusion}

In this dissertation, we present optical characterization and modelling of the Simons Observatory.  Additional analysis is done on ACT DR6 point sources, characterizing the beam of the telescope, which determines the.....

\section{Optical Characterization of Materials}

Understanding the optical properties of the materials within the telescope is crucial for predicting its optical performance.  This becomes even more critical as detectors increase in sensitivity; optical systematics need to be tightly constrained.

In this work, I presented the characterization of a variety of optical components for ground-based cosmological experiments.  Chapter~\ref{ch:si} reports the measured reflectivity of two silicon samples using a holography imaging setup.  From the reflectivity, we determine the loss-tangent of both samples, and find the "intrinsic" silicon has a lower loss-tangent across the mid-frequency band (75-160\,GHz).  In Chapter~\ref{ch:mma} we use the same holography receiver to characterize and develop meta-material absorbers which control scattered light inside the Simons Observatory optics tubes.  In addition to reflectivity, we measure the scattering properties of the meta-material absorbers. 

\section{Optical Characterization of Telescopes}
\subsection{The Simons Observatory}
Integrating and testing the optical performance of a telescope prior to deployment is novel; it is common to characterize the beam of a telescope at the site.  However, once at the site, it is nearly impossible to alter the hardware and probe its functionality.  In Chapter~\ref{ch:ot_holo}, we present the characterization of the Simons Observatory Large Aperture Telescope Receiver with radio holography.

\subsection{The Atacama Cosmology Telescope}