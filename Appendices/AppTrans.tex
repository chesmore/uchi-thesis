\chapter{Harmonic Transform of the On-Sky Beam}
\label{app:trans} 

Here we show the derivation to convert our beam $B(\theta)$ to its spherical transform $b_\ell$. The spherical harmonics are functions defined on the sphere as
\begin{equation}
    Y_{\ell m}(\hat{n}) = P_{\ell m}(\cos\theta)\exp{i m\phi} \; ,
\end{equation}
where $P_{\ell m}$ are the associated Legendre polynomials, $\ell$ and $m$ are multipoles, and $\theta$ and $\phi$ are standard sky coordinates. The associated Legendre polynomials are defined in terms of Legendre polynomials as:
\begin{equation}
    P_{\ell m}(x) = (-1)^m\sqrt{\frac{2\ell+1}{4\pi}}\sqrt{\frac{(\ell-\abs{m})!}{(\ell+\abs{m})!}}(1-x^2)^{\abs{m}/2}\frac{d^{\abs{m}}}{dx^{\abs{m}}}P_\ell(x) \; ,
\end{equation}  
The spherical harmonics are orthonormal, and thus obey the relation
\begin{equation}
    \int_0^{2\pi}\int_0^{\pi} Y_{\ell m}(\hat{n})Y_{\ell'm'}^{*}(\hat{n}) \sin\theta \; d\theta\; d\phi = \delta_{\ell\ell'}\delta_{mm'} \; ,
\end{equation}
where $\overline{z}$ is the complex conjugate of $z$ and $\delta_{ij}$ is the Kronecker symbol.

In a spherical harmonic transform, we compute the coefficients $f_{\ell m}$ used to express a function $f (\hat{n})$ as
\begin{equation}
    f(\hat{n}) = \sum_{\ell=0}^{\infty}\sum_{m=-\ell}^{\ell}f_{\ell m} Y_{\ell m}(\hat{n}) \; .
\end{equation} 
The coefficients can be computed using the equation 
\begin{subequations}
\begin{align}
    f_{\ell m} &= \int_{0}^{2\pi}\int_{0}^{\pi} f(\hat{n}) Y_{\ell m}^{*}(\hat{n})\sin\theta \; d\theta\; d\phi   \\
    & = \int_{0}^{2\pi}\int_{0}^{\pi} f(\hat{n}) P_{\ell m}(\cos\theta)\exp{-im\phi} \sin\theta\; d\theta\; d\phi \; .
\end{align}
\end{subequations}
If $f (\hat{n})$ is independent of $\phi$ (as is approximately true for our beam), then we can write $f (\hat{n})$ = $f (\theta)$ and the equation above becomes
\begin{equation}
    f_{\ell m}= \int_{0}^{2\pi}\exp{-im\phi}d\phi\int_{0}^{\pi} f(\theta) P_{\ell m}(\cos\theta) \sin\theta\; d\theta \; .
\end{equation}
The integral over $\phi$ then simplifies to 
\begin{equation}
    \int_0^{2\pi} e^{-im\phi}\; d\phi = 2\pi \delta_{m0}\; .
\end{equation}
So $f_{\ell m}$ is only non-zero for $m=0$, in which case we have 
\begin{subequations}
\begin{align}
    f_{\ell 0} & =  2\pi\int_0^{\pi}f(\theta)P_{\ell 0}(\cos\theta)\sin\theta \; d\theta\\
    & = 2\pi \int_{-1}^{1}f(\theta)P_{\ell 0} (\cos\theta) \; d\cos \theta\; .
\end{align}
\end{subequations}
This is the equation for the spherical harmonic coefficients of an axisymmetric function $f(\theta)$. The coefficients $F_\ell$ of the Legendre polynomial transform of $f(\theta)$ are related to the spherical harmonic coefficients as:

\begin{equation}
    F_\ell = f_{\ell m} \sqrt{\frac{4\pi}{2\ell+1}}\; .
\end{equation}

The beam transform $B_\ell$ that is used to describe the convolution of the sky with the telescope’s beam is defined as the Legendre polynomial transform of the beam function $B(\theta)$, see e.g. Eq. 48 in \cite{challinor_2000}."