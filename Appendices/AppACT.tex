\chapter{Polarized Beam Transformation Formalism} % Main appendix title
\label{app:act} 

Here, we derive the locally defined map-space polarization basis to determine the polarized $\ell$-space beams from the $Q$ and $U$ beams.  The basis of $Q_r$ and $U_r$ are defined as local linear combinations of the $Q$ and $U$ maps as:
\begin{equation} \label{eq:q_r}
\begin{split}
    Q_r(\boldsymbol{\theta}) =& Q(\boldsymbol{\theta}) \cos 2\phi_{\theta} + U(\boldsymbol{\theta}) \sin 2\phi_{\theta} \\
    U_r(\boldsymbol{\theta}) =& U(\boldsymbol{\theta}) \cos 2\phi_{\theta} - Q(\boldsymbol{\theta}) \sin 2\phi_{\theta}
\end{split}
\end{equation}
where $\boldsymbol{\theta} \equiv (\theta,\phi_\theta)$ are standard polar coordinates with the beam's center as their origin and $\phi_{\theta}$ increasing clockwise from the positive $y$-axis (assuming $+x$ points to the right and $+y$ points upward).  We can also define these fields conversely:
\begin{equation} \label{eq:q}
\begin{split}
    Q(\boldsymbol{\theta}) =& Q_r(\boldsymbol{\theta}) \cos 2\phi_{\theta} - U_r(\boldsymbol{\theta}) \sin 2\phi_{\theta} \\
    U(\boldsymbol{\theta}) =& U_r(\boldsymbol{\theta}) \cos 2\phi_{\theta} + Q_r(\boldsymbol{\theta}) \sin 2\phi_{\theta} \; .
\end{split}
\end{equation}
In order to determine beam leakage in the angular power spectra, we need to translate the polarized beams $Q_r$ and $U_r$ to the $\ell$-space $E$ and $B$ polarized beams.  To do so, we employ the relation between the azimuthally averaged versions of each component, derived here in the flat-sky limit.  The Fourier-space expressions for $E$ and $B$ are:
\begin{equation} \label{eq:e}
\begin{split}
        E(\boldsymbol{\ell}) =& \hat{Q}(\boldsymbol{\ell}) \cos 2\phi_{\ell} + \hat{U}(\boldsymbol{\ell}) \sin 2\phi_{\ell}\\
        B(\boldsymbol{\ell}) =& \hat{U}(\boldsymbol{\ell}) \cos 2\phi_{\ell} - \hat{Q}(\boldsymbol{\ell}) \sin 2\phi_{\ell}
\end{split}
\end{equation}
where $\boldsymbol{\ell} \equiv (\ell,\phi_\ell)$ is the Fourier conjugate of $\boldsymbol{\theta}$, and $\{\hat{Q},\hat{U}\}$ are the standard Fourier transforms of $\{Q,U\}$:
\begin{equation} \label{eq:q_hat}
\begin{split}
        \hat{Q}(\boldsymbol{\ell}) =& \int Q(\boldsymbol{\theta}) e^{i\boldsymbol{\ell}\cdot\boldsymbol{\theta}} d\boldsymbol{\theta} = \int Q(\theta,\phi_{\theta}) e^{i\ell\theta\cos(\phi_{\theta} - \phi_\ell)} \theta\; d\theta\; d\phi_{\theta} \\
        \hat{U}(\boldsymbol{\ell}) =& \int U(\boldsymbol{\theta}) e^{i\boldsymbol{\ell}\cdot\boldsymbol{\theta}} d\boldsymbol{\theta} = \int U(\theta,\phi_{\theta}) e^{i\ell\theta\cos(\phi_{\theta} - \phi_{\ell})} \theta\; d\theta\; d\phi_{\theta} \; .
\end{split}
\end{equation}

Taking the azimuthal average of Equations \ref{eq:e} gives the one-dimensional transforms $\tilde{E}$ and $\tilde{B}$:
\begin{equation} \label{eq:tilde_e1}
\begin{split}
    \tilde{E}(\ell) =& \frac{1}{2\pi} \int \hat{Q}(\boldsymbol{\ell}) \cos 2\phi_{\ell}\;d\phi_{\ell} + \frac{1}{2\pi} \int \hat{U}(\boldsymbol{\ell}) \sin 2\phi_{\ell}\;d\phi_{\ell}   \\
    \tilde{B}(\ell) =& \frac{1}{2\pi} \int \hat{U}(\boldsymbol{\ell}) \cos 2\phi_{\ell}\;d\phi_{\ell} - \frac{1}{2\pi} \int \hat{Q}(\boldsymbol{\ell}) \sin 2\phi_{\ell}\;d\phi_{\ell} \; .    
\end{split}
\end{equation}
We can re-write the expression for $\tilde{E}$ in Equation \ref{eq:tilde_e1} in terms of map-space $Q$ and $U$ by using Equations~\ref{eq:q_hat} (from here on we derive only for $E$ but $B$ can be derived identically):
\begin{equation} \label{eq:tilde_e2}
\begin{split}
        \tilde{E}(\ell) = & \frac{1}{2\pi} \int Q(\theta,\phi_{\theta}) e^{i\ell\theta\cos(\phi_{\theta} - \phi_{\ell})} \theta\;d\theta\;d\phi_{\theta} \cos 2\phi_{\ell}\;d\phi_{\ell} \\ &
    + \frac{1}{2\pi} \int U(\theta,\phi_{\theta}) e^{i\ell\theta\cos(\phi_{\theta} - \phi_{\ell})} \theta\;d\theta\;d\phi_{\theta} \sin 2\phi_{\ell}\;d\phi_{\ell} \; .
\end{split}
\end{equation}

Substituting for $Q(\boldsymbol{\theta})$ and $U(\boldsymbol{\theta})$ in Equation \ref{eq:tilde_e2} using Equations in \ref{eq:q}, we find the relation between $\tilde{E}$ and $\{Q_r,U_r\}$:
\begin{equation} \label{eq:tilde_e3}
\begin{split}
    \tilde{E}(\ell) = & \frac{1}{2\pi} \int \big(Q_r(\theta,\phi_{\theta})\cos 2\phi_{\theta} - U_r(\theta,\phi_{\theta})\sin 2\phi_{\theta}\big) e^{i\ell\theta\cos(\phi_{\theta} - \phi_{\ell})} \theta\;d\theta\;d\phi_{\theta} \cos 2\phi_{\ell}\;d\phi_{\ell} \\ & 
    + \frac{1}{2\pi} \int \big(U_r(\theta,\phi_{\theta})\cos 2\phi_{\theta} + Q_r(\theta,\phi_{\theta})\sin 2\phi_{\theta}\big) e^{i\ell\theta\cos(\phi_{\theta} - \phi_{\ell})} \theta\;d\theta\;d\phi_{\theta} \sin 2\phi_{\ell}\;d\phi_{\ell}\; .
\end{split}
\end{equation}
Now grouping terms together with $Q_r(\boldsymbol{\theta})$ and $U_r(\boldsymbol{\theta})$, we re-write the above equation and find:
\begin{equation} \label{eq:tilde_e4}
\begin{split}
    \tilde{E}(\ell) = & \frac{1}{2\pi} \int Q_r(\theta,\phi_{\theta})\big(\cos 2\phi_{\theta} \cos 2\phi_{\ell} + \sin 2\phi_{\theta} \sin 2\phi_{\ell}\big) e^{i\ell\theta\cos(\phi_{\theta} - \phi_{\ell})} \theta\;d\theta\;d\phi_{\theta}\;d\phi_{\ell} \\ &
    + \frac{1}{2\pi} \int U_r(\theta,\phi_{\theta})\big(\cos 2\phi_{\theta} \sin 2\phi_{\ell} - \sin 2\phi_{\theta} \cos 2\phi_{\ell}\big) e^{i\ell\theta\cos(\phi_{\theta} - \phi_{\ell})} \theta\;d\theta\;d\phi_{\theta}\;d\phi_{\ell}\; .
\end{split}
\end{equation}
Then, making use of the trigonometric identity:
\begin{equation}
    \begin{split}
        \cos(\alpha\pm\beta) =& \cos\alpha\cos\beta \mp \sin\alpha\sin\beta \\
        \sin(\alpha\pm\beta) =& \sin\alpha\cos\beta \mp \cos\alpha\sin\beta
    \end{split}
\end{equation}
...Equation \ref{eq:tilde_e4} may be re-written as:
\begin{equation} \label{eq:tilde_e5}
\begin{split}
    \tilde{E}(\ell) = & \frac{1}{2\pi} \int Q_r(\theta,\phi_{\theta})\cos 2(\phi_{\theta}-\phi_{\ell})\;e^{i\ell\theta\cos(\phi_{\theta} - \phi_{\ell})} \theta\;d\theta\;d\phi_{\theta}\;d\phi_{\ell} \\ &
    - \frac{1}{2\pi} \int U_r(\theta,\phi_{\theta})\sin 2(\phi_{\theta}-\phi_{\ell})\;e^{i\ell\theta\cos(\phi_{\theta} - \phi_{\ell})} \theta\;d\theta\;d\phi_{\theta}\;d\phi_{\ell} \; .
\end{split}
\end{equation}
To simplify this, we define $\phi_{\rho} \equiv \phi_{\theta} - \phi_{\ell}$, and Equation \ref{eq:tilde_e5} becomes:
\begin{equation} \label{eq:tilde_e6}
\begin{split}
    \tilde{E}(\ell) = & \frac{1}{2\pi} \int Q_r(\theta,\phi_{\theta})\cos 2\phi_{\rho}\;e^{i\ell\theta\cos\phi_{\rho}}\;\theta\;d\theta\;d\phi_{\theta}\;d\phi_{\rho}\\ &
    - \frac{1}{2\pi} \int U_r(\theta,\phi_{\theta})\sin 2\phi_{\rho}\;e^{i\ell\theta\cos\phi_{\rho}}\;\theta\;d\theta\;d\phi_{\theta}\;d\phi_{\rho}\; .
\end{split}
\end{equation}
Each of the two terms in Equation~\ref{eq:tilde_e6} can be expressed as three separate integrals:
\begin{equation} \label{eq:sep_int}
\begin{split}
    \tilde{E}(\ell) = & \int \big(\frac{1}{2\pi} \int Q_r(\theta,\phi_{\theta})\,d\phi_{\theta}\big) \big(\int \cos 2\phi_{\rho}\;e^{i\ell\theta\cos\phi_{\rho}}\;d\phi_{\rho}\big) \theta\;d\theta \\ & - \int \big(\frac{1}{2\pi} \int U_r(\theta,\phi_{\theta})\;d\phi_{\theta}\big) \big(\int \sin 2\phi_{\rho}\;e^{i\ell\theta\cos\phi_{\rho}}\;d\phi_{\rho}\big) \theta\;d\theta \; .
\end{split}
\end{equation}
Conveniently, the integrals of $Q_r(\boldsymbol{\theta})$ and $U_r(\boldsymbol{\theta})$ over $\phi_{\theta}$ are the azimuthal averages of $Q_r$ and $U_r$.  The integrals over $\phi_{\rho}$ - the second set of parentheses - turn out to have simple analytic counterparts:
\begin{equation}
    \int \cos 2\phi_{\rho}\;e^{i\ell\theta\cos\phi_{\rho}}\;d\phi_{\rho} = -2\pi J_2(\ell\theta)
\end{equation}
\begin{equation}
    \int \sin 2\phi_{\rho}\;e^{i\ell\theta\cos\phi_{\rho}}\;d\phi_{\rho} = 0 \; .
\end{equation}
Plugging these into our expressions for $\tilde{E}(\ell)$, the azimuthally averaged $\ell$-space $E$ beam is simply the second-order Hankel transform of the azimuthally averaged map-space $Q_r$ beam:
\begin{equation} \label{eq:final1}
    \tilde{E}(\ell) = -2\pi\int \tilde{Q}_r(\theta) J_2(\ell\theta)\;\theta\;d\theta \; .
\end{equation}
While the same relation exists between the azimuthally averaged $\ell$-space $B$ and map-space $U_r$ beams:
\begin{equation} \label{eq:final2}
    \tilde{B}(\ell) = -2\pi\int \tilde{U}_r(\theta) J_2(\ell\theta)\;\theta\;d\theta \; .
\end{equation}
The final Equations \ref{eq:final1} and \ref{eq:final2} provide a complete formalism for transforming the polarized beams using the $\{Q_r,U_r\}$ basis.
